\documentclass[12pt]{article}

%opening
\title{Data Science}
\author{Kevin O'Brien}

\begin{document}
	\large
\subsection*{Supervised and Unsupervised Learning}
\textbf{Supervised learning} is tasked with learning a function from labeled training data in order to predict the value of any valid input. 

Common examples of supervised learning include classifying e-mail messages as spam, labeling Web pages according to their genre, and recognizing handwriting. Many algorithms are used to create supervised learners, the most common being neural networks, Support Vector Machines (SVMs), and Naive Bayes classifiers.
\\
\textbf{Unsupervised learning} is tasked with making sense of data without any examples of what is correct or incorrect. It is most commonly used for clustering similar input into logical groups. Unsupervised learning  can be used to reduce the number of dimensions in a data set in order to focus on only the most useful attributes, or to detect trends. 

Common approaches to unsupervised learning include k-Means, hierarchical clustering, and self-organizing maps.
\end{document}