Binary logistic regression diagnostics
Learn more about Minitab 17 
After any modeling procedure, you typically validate the model. Stat > Regression > Binary Logistic Regression > Fit Binary Logistic Model has a collection of diagnostic plots, goodness-of-fit tests, and other diagnostic measures to validate the model. Residuals and other diagnostic statistics help to find the following potential problems:
Factor/covariate patterns that do not have an acceptable fit
Factor/covariate patterns that have a strong effect on the parameter estimates
Factor/covariate patterns that have a large leverage
Minitab provides different options for each of these potential problems, as listed in the following table. Hosmer and Lemeshow1 indicate that you interpret these diagnostics jointly to understand any potential problems with the model.

Potential problem	Diagnostic statistic	Definition of statistic
Factor/covariate patterns that do not have an acceptable fit	Pearson residual	The difference between the actual and predicted observation
Standardized Pearson residual	The difference between the actual and predicted observation, but standardized to have σ = 1
Deviance residual	Deviance residuals, a component of deviance chi-square
Delta chi-square	Changes in the Pearson chi-square when the jth factor/covariate pattern is removed
Delta deviance	Changes in the deviance when the jth factor/covariate pattern is removed
Factor/covariate patterns that have a strong effect on the parameter estimates	Delta beta calculated with the Pearson residuals	Changes in the coefficients when the jth factor/covariate pattern is removed
Delta beta calculated with the standardized Pearson residuals	Changes in the coefficients when the jth factor/covariate pattern is removed
Factor/covariate patterns that have a large leverage	Leverage (Hi)	Leverages of the jth factor/covariate pattern, a measure of how unusual predictor values are
Residual plots let you visualize some of these diagnostics. You can also store and plot other diagnostics. Delta chi-square and delta deviance are useful for identifying factor/covariate patterns that do not fit the model well. The delta beta statistics are useful for identifying a factor/covariate pattern with a strong effect on the parameter estimates. Typically, you plot these delta statistics against either the estimated event probability or the leverage. The estimated event probability is the probability of the event, given the data and model. Leverages are used to assess how unusual the predictor values are. You can use Minitab's graph brushing capabilities to identify points that are on a graph.
