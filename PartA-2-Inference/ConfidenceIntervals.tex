
 

 

Confidence Intervals

Calculation of upper and lower limits 

We are initially asked to use a particular Confidence level (1-α)%  

From it we deduce the Significance level required   (α)%  

 is then determined from the tables

 

Example letting the confidence level be 95%, we immediately deduce that the

Significance level is 5%

 

(1-α/2) is therefore 97.5%

From the tables we see the associated Z value is 1.96

 

 

 

 

The (1-α)%  [95%]confidence interval means the true value of the parameter (i.e the mean) lies within our confidence interval with a probability of 95%.

 

The remaining α%  probability is distributed on either side of this confidence interval. In other words, the (1-α)%  confidence interval occurs within between the (α/2)% quantile and the (1-α/2)% quantile.

 

A 95% confidence interval is located between the 2.5% and 97.5% quantiles.

 

Formally the lower limit 

and the Upper limit is 

 

Condensing the calculations

 

We can use the symmetry property , the fact that , to simplify into this form 

 

e.g. 

 

