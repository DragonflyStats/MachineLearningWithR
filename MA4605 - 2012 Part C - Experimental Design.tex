\documentclass[12pt, a4paper]{report}
\usepackage{epsfig}
\usepackage{subfigure}
%\usepackage{amscd}
\usepackage{amssymb}
\usepackage{amsbsy}
\usepackage{amsthm}
%\usepackage[dvips]{graphicx}
\usepackage{natbib}
\bibliographystyle{chicago}
\usepackage{vmargin}
% left top textwidth textheight headheight
% headsep footheight footskip
\setmargins{3.0cm}{2.5cm}{15.5 cm}{22cm}{0.5cm}{0cm}{1cm}{1cm}
\renewcommand{\baselinestretch}{1.5}
\pagenumbering{arabic}
\theoremstyle{plain}
\newtheorem{theorem}{Theorem}[section]
\newtheorem{corollary}[theorem]{Corollary}
\newtheorem{ill}[theorem]{Example}
\newtheorem{lemma}[theorem]{Lemma}
\newtheorem{proposition}[theorem]{Proposition}
\newtheorem{conjecture}[theorem]{Conjecture}
\newtheorem{axiom}{Axiom}
\theoremstyle{definition}
\newtheorem{definition}{Definition}[section]
\newtheorem{notation}{Notation}
\theoremstyle{remark}
\newtheorem{remark}{Remark}[section]
\newtheorem{example}{Example}[section]
\renewcommand{\thenotation}{}
\renewcommand{\thetable}{\thesection.\arabic{table}}
\renewcommand{\thefigure}{\thesection.\arabic{figure}}
\title{MA4605}
\author{ } \date{ }


\begin{document}
\author{Kevin O'Brien}
\title{MA4605}

\tableofcontents \setcounter{tocdepth}{2}
%-------------------------------------------------

\chapter{Chemometrics}


\section{Statistical Assumptions}
\section{Testing Normality}
An assessment of the normality of data is a prerequisite for many statistical tests as normal data is an underlying assumption in parametric testing. There are two main methods of assessing normality - graphically and numerically.

\subsection{Shapiro-Wilk}
Thee Shapiro-Wilk Test is more appropriate for small sample sizes (i.e. $n < 50$) but can also handle sample sizes as large as 2000.

\begin{itemize}
\item[H0] The data are not normally distributed
\item[H1] The data are nor normally distributed
\end{itemize}

\begin{verbatim}
Anderson-Darling normality test
data:  x
A = 0.3228, p-value = 0.4624
\end{verbatim}
%------------------------------------------------------------------------%
\chapter{Linear Models}
\section{Multiple Linear Regression}
\section{Variable Selection Procedures}


\begin{itemize}
\item Akaike Information Criterion
\item Multicollinearity
\end{itemize}
%------------------------------------------------------------------------%
\chapter{Design of Experiments}

\section{factorial design}

\section{Latin Square Design}

\begin{centering}
\begin{tabular}{|c|c|c|c|c|}
  \hline
  % after \\: \hline or \cline{col1-col2} \cline{col3-col4} ...
  A & B & C & D & E \\
  \hline
\end{tabular}
\end{centering}

\begin{quote}
The sediment quality guideline for nickel is a maximum value of 52 ppm.
Two t-tests have been performed to check if the area is significantly contaminated or the difference between the sample mean of 57.56 ppm nickel and the maximum level occurs due to random error. The sample standard deviation of nickel is 19.45 ppm.
\end{quote}

\section{Syllabus}
\begin{itemize}\item translate an experimental description into a statistical
model, including identifying model restrictions and assumptions.
\item develop appropriate hypothesis tests and statistical
comparisons for non-standard de- signs; assess the appropriateness
of computer packages' tests for non-standard designs. \item
communicate experimental designs to a technical audience. \item
evaluate and plan a sound experimental design, including
randomization and power analysis. \item analyze experiments in the
presence of common difficulties, including missing and unbalanced
data.
\end{itemize}

\section{Analysis of Variance}

\section{EVOP}

\section{Randomized Complete Block Designs}

\section{Orthogonal Arrays}

\section{Taguchi's loss function}

\section{Experiments}
Four different mixes of fertiliser have been tested on equal sized
plots of land for their ability to promote crop growth. Also
included in the experiment were plots for which no fertiliser had
been added, acting as a control for comparison. Details of the
five fertiliser treatments, including the control, are as follows,
\\A: no fertiliser added \\B: nitrogen alone \\C: phosphorous \\D:
nitrogen and potassium (standard formulation) \\E: nitrogen and
potassium (special formulation - more expensive) \\\\Set up a
matrix of orthogonal contrasts that can be used to test
comparisons between subgroups of these five fertiliser treatments
that may be of particular interest.





\section{Fisher�s LSD}

\section{CRD power analysis}

\section{Signal to Noise Ratio}
\newpage
The term experimental design refers to a plan for assigning
experimental units to treatment conditions. A good experimental
design serves three purposes.

\begin{itemize}
\item Causation. It allows the experimenter to make causal
inferences about the relationship between independent variables
and a dependent variable.

\item Control. It allows the experimenter to rule out alternative
explanations due to the confounding effects of extraneous
variables (i.e., variables other than the independent variables).

\item Variability. It reduces variability within treatment
conditions, which makes it easier to detect differences in
treatment outcomes.

\end{itemize}

\subsection*{An Experimental Design Example}Consider the following
hypothetical experiment. Acme Medicine is conducting an experiment
to test a new vaccine, developed to immunize people against the
common cold. To test the vaccine, Acme has 1000 volunteers - 500
men and 500 women. The participants range in age from 21 to 70.

In this lesson, we describe three experimental designs - a
completely randomized design, a randomized block design, and a
matched pairs design. And we show how each design might be applied
by Acme Medicine to understand the effect of the vaccine, while
ruling out confounding effects of other factors.

Completely Randomized Design: The completely randomized design is
probably the simplest experimental design, in terms of data
analysis and convenience. With this design, participants are
randomly assigned to treatments.

% Table Placebo Vaccine 500 500 A completely randomized design


In this design, the experimenter randomly assigned participants to
one of two treatment conditions. They received a placebo or they
received the vaccine. The same number of participants (500) were
assigned to each treatment condition (although this is not
required). The dependent variable is the number of colds reported
in each treatment condition. If the vaccine is effective,
participants in the "vaccine" condition should report
significantly fewer colds than participants in the "placebo"
condition.

A completely randomized design relies on randomization to control
for the effects of extraneous variables. The experimenter assumes
that, on average, extraneous factors will affect treatment
conditions equally; so any significant differences between
conditions can fairly be attributed to the independent variable.

Randomized Block Design: With a randomized block design, the
experimenter divides participants into subgroups called blocks,
such that the variability within blocks is less than the
variability between blocks. Then, participants within each block
are randomly assigned to treatment conditions. Because this design
reduces variability and potential confounding, it produces a
better estimate of treatment effects.

% Table Gender  Treatment Placebo Vaccine Male    250 250 Female  250 250


The table to the right shows a randomized block design for the
Acme experiment. Participants are assigned to blocks, based on
gender. Then, within each block, participants are randomly
assigned to treatments. For this design, 250 men get the placebo,
250 men get the vaccine, 250 women get the placebo, and 250 women
get the vaccine.

It is known that men and women are physiologically different and
react differently to medication. This design ensures that each
treatment condition has an equal proportion of men and women. As a
result, differences between treatment conditions cannot be
attributed to gender. This randomized block design removes gender
as a potential source of variability and as a potential
confounding variable.

In this Acme example, the randomized block design is an
improvement over the completely randomized design. Both designs
use randomization to implicitly guard against confounding. But
only the randomized block design explicitly controls for gender.

Note 1: In some blocking designs, individual participants may
receive multiple treatments. This is called using the participant
as his own control. Using the participant as his own control is
desirable in some experiments (e.g., research on learning or
fatigue). But it can also be a problem (e.g., medical studies
where the medicine used in one treatment might interact with the
medicine used in another treatment).

Note 2: Blocks perform a similar function in experimental design
as strata perform in sampling. Both divide observations into
subgroups. However, they are not the same. Blocking is associated
with experimental design, and stratification is associated with
survey sampling.



A matched pairs design is a special case of the randomized block
design. It is used when the experiment has only two treatment
conditions; and participants can be grouped into pairs, based on
some blocking variable. Then, within each pair, participants are
randomly assigned to different treatments.

The table to the right shows a matched pairs design for the Acme
experiment. The 1000 participants are grouped into 500 matched
pairs. Each pair is matched on gender and age. For example, Pair 1
might be two women, both age 21. Pair 2 might be two women, both
age 22, and so on.

For the Acme example, the matched pairs design is an improvement
over the completely randomized design and the randomized block
design. Like the other designs, the matched pairs design uses
randomization to control for confounding. However, unlike the
others, this design explicitly controls for two potential lurking
variables - age and gender.

\subsection{Question}

Which of the following statements are true?
\\
\\
I. A completely randomized design offers no control for lurking
variables.\\ II. A randomized block design controls for the
placebo effect. \\III. In a matched pairs design, participants
within each pair receive the same treatment.
\\
(A) I only (B) II only (C) III only (D) All of the above. (E) None
of the above.
\\
Solution
\\
The correct answer is (E). In a completely randomized design,
experimental units are randomly assigned to treatment conditions.
Randomization provides some control for lurking variables. By
itself, a randomized block design does not control for the placebo
effect. To control for the placebo effect, the experimenter must
include a placebo in one of the treatment levels. In a matched
pairs design, experimental units within each pair are assigned to
different treatment levels.


\section{Placebo}
In an experiment, subjects respond differently after they receive
a treatment, even if the treatment is neutral. A neutral treatment
that has no "real" effect on the dependent variable is called a
placebo, and a subject's positive response to a placebo is called
the placebo effect.

To control for the placebo effect, researchers often administer a
neutral treatment (i.e., a placebo) to the control group. The
classic example is using a sugar pill in drug research. The drug
is effective only if subjects who receive the drug have better
outcomes than subject who receive the sugar pill.


\section{Definitions}
\subsection{Experimentation}

An experiment deliberately imposes a treatment on a group of
objects or subjects in the interest of observing the response.
This differs from an observational study, which involves
collecting and analyzing data without changing existing
conditions. Because the validity of a experiment is directly
affected by its construction and execution, attention to
experimental design is extremely important.



\subsection{Treatment}

In experiments, a treatment is something that researchers administer to experimental units. For example, a corn field is divided into four, each part is 'treated' with a different fertiliser to see which produces the most corn; a teacher practices different teaching methods on different groups in her class to see which yields the best results; a doctor treats a patient with a skin condition with different creams to see which
is most effective.  Treatments are administered to experimental
units by `level', where level implies amount or magnitude. For
example, if the experimental units were given 5mg, 10mg, 15mg of a
medication, those amounts would be three levels of the treatment.


\subsection{Factors}
A factor of an experiment is a controlled independent variable; a variable whose levels are set by the experimenter. A factor is a general type or category of treatments. Different treatments constitute different levels of a factor. For example, three different groups of runners are subjected to different training
methods. The runners are the experimental units, the training methods, the treatments, where the three types of training methods constitute three levels of the factor `type of training'.


\section{Randomization}

Because it is generally extremely difficult for experimenters to eliminate bias using only their expert judgment, the use of randomization in experiments is common practice. In a randomized
experimental design, objects or individuals are randomly assigned
(by chance) to an experimental group. Using randomization is the
most reliable method of creating homogeneous treatment groups,
without involving any potential biases or judgments. There are
several variations of randomized experimental designs, two of
which are briefly discussed below. \subsection{Completely
Randomized Design}

In a completely randomized design, objects or subjects are
assigned to groups completely at random. One standard method for
assigning subjects to treatment groups is to label each subject,
then use a table of random numbers to select from the labelled
subjects. This may also be accomplished using a computer. In
MINITAB, the "SAMPLE" command will select a random sample of a
specified size from a list of objects or numbers.
\subsection{Randomized Block Design}

If an experimenter is aware of specific differences among groups
of subjects or objects within an experimental group, he or she may
prefer a randomized block design to a completely randomized
design. In a block design, experimental subjects are first divided
into homogeneous blocks before they are randomly assigned to a
treatment group. If, for instance, an experimenter had reason to
believe that age might be a significant factor in the effect of a
given medication, he might choose to first divide the experimental
subjects into age groups, such as under 30 years old, 30-60 years
old, and over 60 years old. Then, within each age level,
individuals would be assigned to treatment groups using a
completely randomized design. In a block design, both control and
randomization are considered.

\subsection{Example}

A researcher is carrying out a study of the effectiveness of four
different skin creams for the treatment of a certain skin disease.
He has eighty subjects and plans to divide them into 4 treatment
groups of twenty subjects each. Using a randomized block design,
the subjects are assessed and put in blocks of four according to
how severe their skin condition is; the four most severe cases are
the first block, the next four most severe cases are the second
block, and so on to the twentieth block. The four members of each
block are then randomly assigned, one to each of the four
treatment groups.
%------------------------------------------------------------------------%
\chapter{Appendix}
\section{: Glossary of terms}
\begin{itemize}
\item Parts Per Million (denoted as `ppm')
\end{itemize}

%------------------------------------------------%
%\newpage
%\addcontentsline{toc}{section}{Bibliography}
%\bibliography{MA4125bib}
\end{document} 