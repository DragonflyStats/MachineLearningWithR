%% - http://www.statstutor.ac.uk/resources/uploaded/multilevelmodelling.pdf

Statistics: Multilevel modelling
Richard Buxton. 2008.

\subsection{1 Introduction}
\begin{itemize}
\item Multilevel modelling is an approach that can be used to handle clustered or grouped data.
Suppose we are trying to discover some of the factors that affect a child’s academic
attainment in English at age 16. The sample of pupils involved in our study will be
taught in classes, within schools. We’ll probably be interested in the effect of a mix of
pupil level factors - e.g.the socioeconomic status of the child’s parents, class level factors -
e.g. the use of streamed vs unstreamed teaching, and school level factors - e.g. single-sex
vs mixed. Multi-level modelling provides a useful framework for thinking about problems
with this type of hierarchical structure.
\item Even if we are mainly interested in pupil level factors, we’ll still need to take account of
the clustering in our sample. For example, the attainment levels of two children in the
same class will tend to be more similar than the levels of two children in different classes.
If we use statistical techniques that ignore the clustering - e.g. multiple regression - the
standard errors and confidence intervals that we obtain will be unrealistic and we may
well conclude that there are real effects, when we are simply looking at random variation.
Multilevel modelling can also be used to analyse repeated measures data. For example,
if we are measuring the blood pressure of a group of patients at weekly intervals, we can
think of the successive measurements as grouped within the individual subjects. One
advantage of the multilevel modelling approach is that it can deal with data in which the
times of the measurements vary from subject to subject.
\item The aim of this handout is to introduce the idea of multilevel modelling. If you feel that
this approach is likely to be helpful to you, you may well find it useful to look at some of
the material listed in Section 7.
\end{itemize}
%----------------------------------------------------------------------------------------%
\newpage
2 Example
We illustrate the idea of multilevel modelling with a set of repeated measures data giving
growth patterns for a sample of 26 boys in Oxford, England1
- see Figure 1. The height
of each boy is measured on nine different occasions. The plots for the individual boys run
from the bottom left to the top right and are arranged in order of the maximum height
1Source of data: Pinheiro and Bates (2000)
1
attained over the period of measurement.
Centered age
Height (cm)
130
140
150
160
170
−1.0 0.5
10 26
−1.0 0.5
25 9
−1.0 0.5
2 6
−1.0 0.5
7 17
−1.0 0.5
16 15
−1.0 0.5
8 20
−1.0 0.5
1
18
−1.0 0.5
5 23
−1.0 0.5
11 21
−1.0 0.5
3 24
−1.0 0.5
22 12
−1.0 0.5
13 14
−1.0 0.5
19
130
140
150
160
170
4
Figure 1: Growth patterns of boys in Oxford
For each boy, the growth pattern appears to be roughly linear, so we might try modelling
it by a simple linear regression model of the form. . .
H = β0 + β1A + ² (1)
. . . where H and A represent height and age and ² represents the variation in height that
cannot be explained by the linear relationship with age.
To extend our model beyond a single boy, we need to allow for the variation in growth
patterns among different subjects. For example, a quick glance at Figure 1 shows that
some of the subjects are consistently taller than others - compare subject 4 in the top
right of the plot with subject 10 in the bottom left. If we try to use Model 1 for the
complete set of data, the fit will be very poor - see Figure 2.
2
Centered age
Height (cm)
130
140
150
160
170
−1.0 0.5
10 26
−1.0 0.5
25 9
−1.0 0.5
2 6
−1.0 0.5
7 17
−1.0 0.5
16 15
−1.0 0.5
8 20
−1.0 0.5
1
18
−1.0 0.5
5 23
−1.0 0.5
11 21
−1.0 0.5
3 24
−1.0 0.5
22 12
−1.0 0.5
13 14
−1.0 0.5
19
130
140
150
160
170
4
Figure 2: Simple linear regression model fitted to complete dataset
To make our model more realistic, we allow the intercept in Model 1 to vary from subject
to subject. Writing Hij for the ith measurement on the jth subject, we have . . .
Hij = β0j + β1Aij + ²ij (2)
Notice that the intercept β0j now has a subscript j, indicating that it will vary from
subject to subject.
We now assume that the individual intercepts follow a Normal distribution with variance
τ0. This gives the model. . .
β0j = β0 + u0j (3)
3
. . . where u0j ∼ N(0, τ0)
Model 2 accounts for the variation in the individual measurements on a single subject,
while Model 3 accounts for the variation from one subject to another. The combination
of these two models gives what is known as a multilevel model.
Fitting our multilevel model to the data in Figure 1, we obtain the predictions shown in
Figure 3.
Centered age
Height (cm)
130
140
150
160
170
−1.0 0.5
10 26
−1.0 0.5
25 9
−1.0 0.5
2 6
−1.0 0.5
7 17
−1.0 0.5
16 15
−1.0 0.5
8 20
−1.0 0.5
1
18
−1.0 0.5
5 23
−1.0 0.5
11 21
−1.0 0.5
3 24
−1.0 0.5
22 12
−1.0 0.5
13 14
−1.0 0.5
19
130
140
150
160
170
4
Figure 3: Multilevel model with varying intercept
The fit is much better than for the simple linear regression model, but there is still some
evidence of lack of fit. Although we’re allowing the intercept to vary from subject to
subject, we’re using a common slope. As a result, we’re overestimating the growth rate
of some subjects - e.g. subject 10, and underestimating the growth rate of others - e.g.
subjects 4 or 19. Perhaps we can improve our model by allowing both the intercepts and
slopes to vary randomly. Figure 4 shows the predictions from a model of this kind.
4
Centered age
Height (cm)
130
140
150
160
170
−1.0 0.5
10 26
−1.0 0.5
25 9
−1.0 0.5
2 6
−1.0 0.5
7 17
−1.0 0.5
16 15
−1.0 0.5
8 20
−1.0 0.5
1
18
−1.0 0.5
5 23
−1.0 0.5
11 21
−1.0 0.5
3 24
−1.0 0.5
22 12
−1.0 0.5
13 14
−1.0 0.5
19
130
140
150
160
170
4
Figure 4: Multilevel model with varying intercept and slope
3 How do multilevel models differ from regression
models?
To show the difference between a multilevel model and an ordinary regression model, we
return to the model with varying intercepts and substitute equation 3 into equation 2 to
give. . .
Hij = (β0 + u0j ) + β1Aij + ²ij = β0 + β1Aij + u0j + ²ij (4)
The feature that distinguishes this model from an ordinary regression model is the presence
of two random variables - the measurement level random variable ²ij and the subject level
random variable u0j
.
Because multilevel models contain a mix of fixed effects and random effects, they are
5
sometimes known as mixed-effects models.
4 Benefits of multilevel modelling
In a multilevel model, we use random variables to model the variation between groups. An
alternative approach is to use an ordinary regression model, but to include a set of dummy
variables to represent the differences between the groups. The multilevel approach offers
several advantages.
• We can generalize to a wider population
– e.g. can say something about the growth curves that we might expect in the
population of boys from which our sample was selected
• Fewer parameters are needed
– With the height data, we only needed one additional parameter - the variance
of the u0,j - in order to allow the intercepts to vary from subject to subject. By
contrast, the approach via dummy variables would require 25 additional parameters.
This reduction in the number of parameters is particularly important
with more complex models and a limited amount of data.
• Information can be shared between groups
– By assuming that the random effects come from a common distribution, a
multilevel model can share information between groups. This can improve the
precision of predictions for groups that have relatively little data.
5 Extending multilevel modelling
This section indicates some of the ways in which multilevel models can be extended to
deal with more complex behaviour.
• Inclusion of predictors at the group level
– e.g. if our height data included both boys and girls, we could include a term
in the group level model to describe the difference in height between boys and
girls
• Multiple levels of grouping
– e.g. pupils within classes within schools
• Multivariate responses
– e.g. exam results in English and Maths at age 16
6
• Cross-classified data
– e.g. children in one school may come from different areas and children from
the same area may go to different schools
• Usual regression extensions
– e.g. multiple predictors, categorical predictors and responses, etc
6 Software for multilevel modelling
There is a wide range of software available for fitting multilevel models - for detailed
reviews, see the website of the Centre for Multilevel modelling at the University of Bristol.
7 References
For a short introduction to multilevel modelling, see Browne and Rasbash (2004). For a
more detailed treatment, see Pinheiro and Bates (2000) or Raudensbush and Bryk (2002).
Both these texts contain extensive discussions of the application of multilevel modelling
to real problems - the examples in Raudensbush and Bryk are taken from the social
and human sciences, while those in Pinheiro and Bates involve areas such as ergonomics,
agriculture and medicine.
Another useful source of information is the website of the Centre for Multilevel Modelling
(CMM) at Bristol University. This contains extensive training material and reviews of
statistical software.
Browne, W. and Rasbash, J. (2004). ‘Multilevel Modelling’, in Hardy, M. and
Bryman, A. (eds.), Handbook of data analysis, Sage Publications, pp 459-78.
Pinheiro, J.C. and Bates, D.M. (2000). Mixed-Effects Models in S and S-PLUS,
Springer.
Raudensbush, S.W. and Bryk, A.S. (2002). Hierarchical Linear Models, Sage
Publications.
