\documentclass[MdouleBmain.tex]{subfiles}

\begin{document}


\subsection{Logarithmic  Transformation}

If data deviate substantially from a Gaussian distribution, using a nonparametric test is not the only alternative. Consider transforming the data to create a Gaussian distribution. Transforming to reciprocals or logarithms are often helpful.
Data can fail a normality test because of the presence of an outlier. Removing that outlier can restore normality.
The decision of whether to use a parametric or nonparametric test is most important with small data sets (since the power of nonparametric tests is so low). But with small data sets, normality tests have little power to detect non-normal distributions, so an automatic approach would give you false confidence.

With large data sets, normality tests can be too sensitive. A low p-value from a normality test tells you that there is strong evidence that the data are not sampled from an ideal normal distribution. But you already know that, as almost no scientifically relevant variables form an ideal normal distribution. What you want to know is whether the distribution deviates enough from the normal ideal to invalidate conventional statistical tests (that assume a Gaussian distribution). A normality test does not answer this question. With large data sets, trivial deviations from the idea can lead to a small p-value.



\section{Non-Parametric Tests}
Many statistical tests assume that you have sampled data from populations that follow a Normal distribution. 
Biological data never follow a Gaussian distribution precisely, because a Gaussian distribution extends infinitely in both directions, and so it includes both infinitely low negative numbers and infinitely high positive numbers. But many kinds of biological data follow a bell-shaped distribution that is approximately Gaussian. 

Because statistical tests work well even if the distribution is only approximately Gaussian (especially with large samples), these tests are used routinely in many fields of science.

An alternative approach does not assume that data follow a Gaussian distribution. These tests, called nonparametric tests, are appealing because they require fewer assumptions about the distribution of the data. In this approach, values are ranked from low to high, and the analyses are based on the distribution of ranks.
Often, the analysis will be one of a series of experiments. Since you want to analyze all the experiments the same way, you cannot rely on the results from a single normality test.


\end{document}
