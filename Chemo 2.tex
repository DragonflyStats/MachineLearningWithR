
\documentclass{article}
%\usepackage{amsmath}
%\accentedsymbol{\dbarA}{\Bar{\Bar{A}}
\title{Transfer}
\author{Kevin O'Brien}
\date{Summer 2009}
\begin{document}

%------------------------------------------------------------------------ %

\[ \bar{X} \]
\[\bar{\bar{X}}  \pm \frac{\bar{s}}{c_4\sqrt{n}} \]


\[ \bar{s} \pm \frac{c_4\bar{s}}{c_4}\]

\[\bar{R}D3, \bar{R}D4\]


%------------------------------------------------------------------------ %
\Large{
\subsection*{Factors for Control Charts}
\begin{tabular}{|c|c|c|c|c|c|c|}
\hline  
Sample Size (n) 	&	c4 	&	c5 	&	d2 	&	d3 	&	D3 	&	D4	\\	\hline
2	&	0.7979	&	0.6028	&	1.128	&	0.853	&	0	&	3.267	\\	
3	&	0.8862	&	0.4633	&	1.693	&	0.888	&	0	&	2.574	\\	
4	&	0.9213	&	0.3889	&	2.059	&	0.88	&	0	&	2.282	\\	
5	&	0.9400	&	0.3412	&	2.326	&	0.864	&	0	&	2.114	\\	
6	&	0.9515	&	0.3076	&	2.534	&	0.848	&	0	&	2.004	\\	
7	&	0.9594	&	0.282	&	2.704	&	0.833	&	0.076	&	1.924	\\	
8	&	0.9650	&	0.2622	&	2.847	&	0.82	&	0.136	&	1.864	\\	
9	&	0.9693	&	0.2459	&	2.970	&	0.808	&	0.184	&	1.816	\\	
10	&	0.9727	&	0.2321	&	3.078	&	0.797	&	0.223	&	1.777	\\	
11	&	0.9754	&	0.2204	&	3.173	&	0.787	&	0.256	&	1.744	\\	
12	&	0.9776	&	0.2105	&	3.258	&	0.778	&	0.283	&	1.717	\\	
13	&	0.9794	&	0.2019	&	3.336	&	0.770	&	0.307	&	1.693	\\	
14	&	0.9810	&	0.1940	&	3.407	&	0.763	&	0.328	&	1.672	\\	
15	&	0.9823	&	0.1873	&	3.472	&	0.756	&	0.347	&	1.653	\\	
16	&	0.9835	&	0.1809	&	3.532	&	0.750	&	0.363	&	1.637	\\
17	&	0.9845	&	0.1754	&	3.588	&	0.744	&	0.378	&	1.622	\\
18	&	0.9854	&	0.1703	&	3.64	&	0.739	&	0.391	&	1.608	\\
19	&	0.9862	&	0.1656	&	3.689	&	0.734	&	0.403	&	1.597	\\
20	&	0.9869	&	0.1613	&	3.735	&	0.729	&	0.415	&	1.585	\\
21	&	0.9876	&	0.1570	&	3.778	&	0.724	&	0.425	&	1.575	\\
22	&	0.9882	&	0.1532	&	3.819	&	0.720	&	0.434	&	1.566	\\
23	&	0.9887	&	0.1499	&	3.858	&	0.716	&	0.443	&	1.557	\\
24	&	0.9892	&	0.1466	&	3.895	&	0.712	&	0.451	&	1.548	\\
25	&	0.9896	&	0.1438	&	3.931	&	0.708	&	0.459	&	1.541	\\
\hline 
\end{tabular} 
} % End Large Font

%\begin{framed}
\begin{verbatim}
> PCs=c(0.90,0.95,0.975,0.99,0.995)
> qnorm(PCs)
[1] 1.281552 1.644854 1.959964 2.326348 2.575829
\end{verbatim}
%\end{framed}
\begin{verbatim}
> dpois(0:10,lambda=2)
 [1] 1.353353e-01 2.706706e-01 2.706706e-01 1.804470e-01 9.022352e-02
 [6] 3.608941e-02 1.202980e-02 3.437087e-03 8.592716e-04 1.909493e-04
[11] 3.818985e-05
> 
> ppois(0:10,lambda=2)
 [1] 0.1353353 0.4060058 0.6766764 0.8571235 0.9473470
 [6] 0.9834364 0.9954662 0.9989033 0.9997626 0.9999535
[11] 0.9999917
>
> dpois(0:10,lambda=0.5)
 [1] 6.065307e-01 3.032653e-01 7.581633e-02 1.263606e-02
 [5] 1.579507e-03 1.579507e-04 1.316256e-05 9.401827e-07
 [9] 5.876142e-08 3.264523e-09 1.632262e-10
>
> ppois(0:10,lambda=0.5)
 [1] 0.6065307 0.9097960 0.9856123 0.9982484 0.9998279
 [6] 0.9999858 0.9999990 0.9999999 1.0000000 1.0000000
[11] 1.0000000
\end{verbatim}
%-----------------------------------------------------------------
\begin{verbatim}
> pbinom(0:10,size=100,prob=0.10)
 [1] 0.0000265614 0.0003216881 0.0019448847 0.0078364871 0.0237110827
 [6] 0.0575768865 0.1171556154 0.2060508618 0.3208738884 0.4512901654
[11] 0.5831555123
>
> pbinom(0:10,size=100,prob=0.01)
 [1] 0.3660323 0.7357620 0.9206268 0.9816260 0.9965677 
 [6] 0.9994655 0.9999289 0.9999918 0.9999992 0.9999999 
[11] 1.0000000
>
> dbinom(0:10,size=100,prob=0.1)
 [1] 0.0000265614 0.0002951267 0.0016231966 0.0058916025 0.0158745955
 [6] 0.0338658038 0.0595787289 0.0888952464 0.1148230266 0.1304162771
[11] 0.1318653468
\end{verbatim}
\newpage
\section{Paired t-test}
\begin{verbatim}
> t.test(After,Before,paired=TRUE)

        Paired t-test

data:  After and Before 
t = 17.4864, df = 7, p-value = 4.924e-07
alternative hypothesis: true difference in means is not equal to 0 
95 percent confidence interval:
 13.40398 17.59602 
sample estimates:
mean of the differences 
                   15.5 
\end{verbatim}
%------------------------------------------------------- %
\section{Question }

factor A has a positive effect on the
response at one level of factor B, while at a different level of factor B the effect of
A is negative. We use the term positive effect here to indicate that the yield or
response increases as the levels of a given factor increase according to some defined
order. In the same sense a negative effect corresponds to a decrease in yield for
increasing levels of the factor.
Consider, for example, the following data on temperature (factor A at levels t\,
t2, and t3 in increasing order) and drying time dx, d2, and d3 (also in increasing
order). The response is percent solids. These data are completely hypothetical
and given to illustrate a point.

\begin{tabular}{|c|c|c|c|c|}
\hline

& d1 &	d2	&	d3	& d4	 \\
t1 & 34, 32.7 &	30.1, 32.8	&	29.8, 26.7	&	29,28.9	 \\
t2 & 32, 33.2 &	30.2, 29.8	&	28.7, 28.1	&	27.6,27.8	\\
t3 & 28.4,29.3	&	27.3,28.9	&	29.7,27.3	&	28.8,29.1	\\

\hline 
\end{tabular} 

\begin{verbatim}
Solid=c(
34.0, 32.7, 32.0, 33.2, 28.4, 29.3, 
30.1, 32.8, 30.2, 29.8, 27.3, 28.9, 
29.8, 26.7, 28.7, 28.1, 29.7, 27.3, 
29.0, 28.9, 27.6, 27.8, 28.8, 29.1)


Solid=c(
34.0, 32.7, 32.0, 33.2, 28.1, 28.0, 
30.1, 32.8, 30.2, 29.8, 27.3, 27.9, 
29.8, 26.7, 28.7, 28.1, 31.1, 31.2, 
29.0, 29.9, 27.6, 27.8, 27.8, 27.1)


Temperature = structure(c(1L, 1L, 1L, 1L, 1L, 1L, 2L, 2L, 2L, 2L, 2L, 2L, 3L, 
3L, 3L, 3L, 3L, 3L, 4L, 4L, 4L, 4L, 4L, 4L), .Label = c("t1", 
"t2", "t3", "t4"), class = "factor")


DryingTime = structure(c(1L, 1L, 2L, 2L, 3L, 3L, 1L, 1L, 2L, 2L, 3L, 3L, 1L, 
1L, 2L, 2L, 3L, 3L, 1L, 1L, 2L, 2L, 3L, 3L), .Label = c("d1", 
"d2", "d3"), class = "factor")

Experiment=data.frame(Solid,Temperature,DryingTime)
interaction.plot(Temperature,DryingTime,Solid)
Model=aov(Solid~Temperature*DryingTime)
summary(Model)
par(font=2)
interaction.plot(DryingTime, Temperature, Solid,col=c("red","blue","black","green"),font.lab=2,
font.axis=2,lty=c(1,2,3,4),lwd=2.5)
\end{verbatim}


\newpage
\begin{verbatim}
> Model1=aov(Solid~Temperature+DryingTime)
> summary(Model2)
            Df Sum Sq Mean Sq F value Pr(>F)  
Temperature  3  30.48  10.162   3.180 0.0491 *
DryingTime   2  17.05   8.525   2.668 0.0967 .
Residuals   18  57.52   3.196                 
---
Signif. codes:  0 ?***? 0.001 ?**? 0.01 ?*? 0.05 ?.? 0.1 ? ? 1 
> 
> Model2=aov(Solid~Temperature*DryingTime)
> summary(Model2)
                       Df Sum Sq Mean Sq F value   Pr(>F)    
Temperature             3  30.48  10.162  10.951 0.000944 ***
DryingTime              2  17.05   8.525   9.188 0.003802 ** 
Temperature:DryingTime  6  46.39   7.732   8.332 0.001026 ** 
Residuals              12  11.13   0.928                     
---
Signif. codes:  0 ?***? 0.001 ?**? 0.01 ?*? 0.05 ?.? 0.1 ? ? 1 
\end{verbatim}
\newpage
\subsection{Statistical Process Control}
A normally distributed quality characteristic is monitored through the use of control charts. These charts have the 
following parameters. All charts are in control.
\begin{tabular}{|c|c|c|c|}
\hline  & LCL & Centre Line & UCL \\ 
\hline $\bar{X}$-Chart & 542 & 550 & 558 \\ 
\hline $R$-Chart & 0 & 8.236 & 16.504 \\ 
\hline 
\end{tabular} 
\begin{itemize}
\item What sample size is being used?
\item Estimate the standard deviation of this process.
\item Compute the control limits for the $s$-chart.
\end{itemize}
\subsection{Multiple Linear Regression}
Explain the term `Multicollinearity'. If Multicollinearity is detected, what are the implications?

\subsection{Experimental Design}
Explain the term `Blocking' in the context of experimental design.

Fill in the missing values in the ANOVA tables. What is the purpose of this inference procedure, stating clearly the null and 
alternative hypothesis.

Explain the following terms in the context of experimental design
factors
levels of factor
blocking 
interaction


FabricWear=c(527, 561, 456, 377, 475, 466, 296, 325, 
621, 664, 426, 476, 561, 540, 301, 235, 
724, 743, 460, 426, 626, 682, 322, 304) 

Filler = structure(c(1L, 1L, 2L, 2L, 1L, 1L, 2L, 2L, 1L, 1L, 2L, 2L, 1L, 
1L, 2L, 2L, 1L, 1L, 2L, 2L, 1L, 1L, 2L, 2L), .Label = c("F1", 
"F2"), class = "factor")

Proportion = structure(c(1L, 1L, 1L, 1L, 1L, 1L, 1L, 1L, 2L, 2L, 2L, 2L, 2L, 
2L, 2L, 2L, 3L, 3L, 3L, 3L, 3L, 3L, 3L, 3L), .Label = c("Q1", 
"Q2", "Q3"), class = "factor")

Surface= structure(c(1L, 1L, 1L, 1L, 2L, 2L, 2L, 2L, 1L, 1L, 1L, 1L, 2L, 
2L, 2L, 2L, 1L, 1L, 1L, 1L, 2L, 2L, 2L, 2L), .Label = c("T0", 
"T1"), class = "factor")

Experiment=data.frame(FabricWear,Filler,Proportion,Surface)

Model1= aov(FabricWear ~ Filler*Proportion*Treatment)

Model2 = aov(FabricWear ~ Filler*Proportion*Treatment - Filler:Proportion:Treatment)


\end{document}
