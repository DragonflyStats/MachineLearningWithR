%- http://www.ats.ucla.edu/stat/spss/dae/manova1.htm
\begin{frame}

\end{frame}
%===================================================================%
\begin{frame}
\frametitle{SPSS : MANOVA}
Examples of one-way multivariate analysis of variance

Example 1. A researcher randomly assigns 33 subjects to one of three groups.  The first group receives technical dietary information interactively from an on-line website.  Group 2 receives the same information from a nurse practitioner, while group 3 receives the information from a video tape made by the same nurse practitioner.  The researcher looks at three different ratings of the presentation, difficulty, usefulness and importance, to determine if there is a difference in the modes of presentation.  In particular, the researcher is interested in whether the interactive website is superior because that is the most cost-effective way of delivering the information.

Example 2.  A clinical psychologist recruits 100 people who suffer from panic disorder into his study.  Each subject receives one of four types of treatment for eight weeks.  At the end of treatment, each subject participates in a structured interview, during which the clinical psychologist makes three ratings:  physiological, emotional and cognitive.  The clinical psychologist wants to know which type of treatment most reduces the symptoms of the panic disorder as measured on the physiological, emotional and cognitive scales.  (This example was adapted from Grimm and Yarnold, 1995, page 246.)

\end{frame}
%===================================================================%
\begin{frame}
\frametitle{SPSS : MANOVA}
Description of the data

Let's pursue Example 1 from above.
We have a data file, manova.sav, with 33 observations on three response variables.  The response variables are ratings called useful, difficulty and importance.  Level 1 of the group variable is the treatment group, level 2 is control group 1 and level 3 is control group 2.

\end{frame}
%===================================================================%
\begin{frame}
\frametitle{SPSS : MANOVA}

Let's look at the data.  It is always a good idea to start with descriptive statistics.

get file='d:\data\manova.sav' .

descriptives
 variables=difficulty useful importance.
 
 
