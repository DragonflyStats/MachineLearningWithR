The runs test (also called Wald–Wolfowitz test after Abraham Wald and Jacob Wolfowitz) is a non-parametric statistical test that checks a randomness hypothesis for a two-valued data sequence. More precisely, it can be used to test the hypothesis that the elements of the sequence are mutually independent.

Under the null hypothesis, the number of runs in a sequence of N elements[1] is a random variable whose conditional distribution given the observation of N+ positive values[2] and N− negative values (N = N+ + N−) is approximately normal, with: [3] [4]
mean \mu=\frac{2\ N_+\ N_-}{N} + 1\,
variance \sigma^2=\frac{2\ N_+\ N_-\ (2\ N_+\ N_--N)}{N^2\ (N-1)}=\frac{(\mu-1)(\mu-2)}{N-1}\,.

%==============================================================================================================================%
THE RUNS TEST FOR RANDOMNESS
Where a run is a series of like observations, the runs test is used to test the randomness of a series of
observations when each observation can be assigned to one of two categories.

%==============================================================================================================================%
EXAMPLE 5. For a random sample of n ¼ 10 individuals, suppose that when they are categorized by sex, the sequence of
observations is: M, M, M, M, F, F, F, F, M, M. For these data there are three runs, or series of like items.
For numeric data, one way by which the required two-category scheme can be achieved is to classify each
observation as being either above or below the median of the group. In general, either too few runs or too many
runs than would be expected by chance would result in rejecting the null hypothesis that the sequence of
observations is a random sequence.

%==============================================================================================================================%
The number of runs of like items is determined for the sample data, with the symbol R used to designate the
number of observed runs. Where n1 equals the number of sampled items of one type and n2 equals the number of
sampled items of the second type, the mean and the standard error associated with the sampling distribution of
the R test statistic when the sequence is random are


\[ Equation 1 \]

\[ Equation 2\]
THE RUNS TEST FOR RANDOMNESS

%=======================================================================================================================================%
17.1. A sample of 36 individuals were interviewed in a market-research survey, with 22 women (W) and 14
men (M) included in the sample. The sampled individuals were obtained in the following order: 

M, W,
W,W,W,M,M,M,W,M,W,W,W,M,M,W,W,W,W,M,W,W,W,M,M,W,W,W,M,W,M,M,
W, W, W, M. 
Use the runs test to test the randomness of this set of observations, using the 5 percent
level of significance.

The number of runs in this sample is R ¼ 17, as indicated by the underscores above. Where n1 ¼ the number of
women and n2 ¼ the number of men, we can use the normal distribution to test the null hypothesis that the sampling
process was random, because n1 . 20. We compute the mean and the standard error of the sampling distribution of
R as follows:
%=======================================================================================================================================%
mR ¼
2n1n2
n1 þ n2 þ 1 ¼
2(22)(14)
22 þ 14 þ 1 ¼
616
36 þ 1 ¼ 18:1
sR ¼

2n1n2(2n1n2 ! n1 ! n2)
(n1 þ n2)2(n1 þ n2 ! 1)
s
¼
%=======================================================================================================================================================================================%
2(22)(14)[2(22)(14) ! 22 ! 14]
(22 þ 14)2(22 þ 14 ! 1)
s
¼
%=======================================================================================%i
616(616 ! 36)
362(35)
s
%=============================================%
357,280
45,360
s
¼
%=======================================%
7:8765 p
¼ 2:81
When the 5 percent level of significance is used, the critical values of the z statistic are z ¼ +1.96. The value
of the z test statistic for these data is
z ¼
R !mR
sR ¼
17 ! 18:1
2:81 ¼ !1:1
2:81 ¼ !0:39

Therefore, using the 5 percent level of significance, we cannot reject the null hypothesis that the sequence of
women and men occurred randomly.
%=======================================================================================================================================%
