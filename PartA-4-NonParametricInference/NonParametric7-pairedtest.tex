PAIRED OBSERVATIONS: THE SIGN TEST
For two samples collected as paired observations (see Section 11.3), the sign test described in Section 17.4
can be used to test the null hypothesis that the two population medians are equal. The sample values must be at
least at the ordinal scale, and no assumptions are required about the forms of the two population distributions.
A plus sign is assigned for each pair of values for which the measurement in the first sample is greater than
the measurement in the second sample, and a minus sign is assigned when the opposite condition is true. If a
pair of measurements have the same value, these tied values are dropped from the analysis, with the effective
sample size thereby being reduced. If the hypothesis that the two populations are at the same level of magnitude
is true, the number of plus signs should approximately equal the number of minus signs. Therefore, the null
hypothesis tested is H0: p ¼ 0.50, where p is the population proportion of the plus (or the minus) signs. If the
sample is large (n . 30), the normal distribution can be used, as described in Section 11.5. Note that although
two samples have been collected, the test is applied to the one set of plus and minus signs that results from the
comparison of the pairs of measurements.
Problem 17.5 illustrates use of the sign test for testing the difference between two medians for data that
have been collected as paired observations.
17.8 PAIRED OBSERVATIONS: THE WILCOXON TEST
For two samples collected as paired observations, the Wilcoxon test described in Section 17.5 can be used
to test the null hypothesis that the two population medians are equal. Because the Wilcoxon test considers the
magnitude of the difference between the values in each matched pair, and not just the direction or sign of the
difference, it is a more sensitive test than the sign test. However, the sample values must be at the interval scale.
No assumptions are required about the forms of the two distributions.
The difference between each pair of values is determined, and this difference, with the associated
arithmetic sign, is designated by d. If any difference is equal to zero, this pair of observations is dropped from
the analysis, thus reducing the effective sample size. Then the absolute values of the differences are ranked from
lowest to highest, with the rank of 1 assigned to the smallest absolute difference.When absolute differences are
equal, the mean rank is assigned to the tied values. Finally, the sum of the ranks is obtained separately for the
positive and negative differences. The smaller of these two sums is the Wilcoxon T statistic for a two-sided test.
For a one-sided test the smaller sum must be associated with the directionality of the null hypothesis, as
illustrated in the one-sample application of the Wilcoxon test in Problem 17.3. Appendix 10 identifies the
critical value of T according to sample size and level of significance. For rejection of the null hypothesis, the T
statistic must be smaller than the critical value given in the table.
When n  25 and the null hypothesis is true, the T statistic is approximately normally distributed. The
formulas for the mean and standard error of the sampling distribution of T and the formula for the z test statistic
are given in Section 17.5, on the one-sample application of the Wilcoxon test.
Problem 17.6 illustrates the use of the Wilcoxon test for testing the difference between two medians data
that have been collected as paired observations.
