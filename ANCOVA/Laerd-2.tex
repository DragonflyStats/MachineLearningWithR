ANCOVA in SPSS Statistics (cont...)

\subsection{SPSS Statistics Output of the ANCOVA}
SPSS Statistics generates quite a few tables in its ANCOVA analysis. In this section, we show you only the main tables required to understand your results from the ANCOVA and post hoc tests. For a complete explanation of the output you have to interpret when checking your data for the nine assumptions required to carry out an ANCOVA, see our enhanced guide. This includes relevant scatterplots and grouped scatterplot, and output from your Shapiro-Wilk test for normality, Levene's test for homogeneity of variances, and tests of between-subjects effects. You can learn more about our enhanced content here.

In the "quick start" guide, we explain the descriptives table, as well as the results for the ANCOVA and post hoc test. We go through each table in turn:

Join the 10,000s of students, academics and professionals who rely on Laerd Statistics.TAKE THE TOUR PLANS & PRICING

%----------------------------------------------------------%
\subsection{Descriptive statistics}
The Descriptive Statistics table (shown below) presents descriptive statistics (mean, standard deviation, number of participants) on the dependent variable, post, for the different levels of the independent variable, group. These values do not include any adjustments made by the use of a covariate in the analysis.


%----------------------------------------------------------%
\subsection{ANCOVA results}
The main section of the results is presented in the Tests of Between-Subjects Effects table, as shown below:


This table informs you whether your ANCOVA was statistically significant. Put another way, whether there was an overall statistically significant difference in post-intervention cholesterol concentration (post) between the different interventions (group) once their means had been adjusted for pre-intervention cholesterol concentrations (pre). This is highlighted below:


Read along the group row until you reach the "Sig." column. This provides the statistical significance value (p-value) of whether there are statistically significant differences between the groups when adjusted for the covariate. In this example, you can see that there is a statistically significant difference between adjusted means (p < .0005).


%----------------------------------------------------------%
\subsection{Estimates}
To get a better understanding of how the covariate has adjusted the original post group means, you can consult the Estimates table, as shown below:


Notice how the mean values have changed compared to those found in the Descriptive Statistics table, above. These new values represent the adjusted means (i.e., the original means adjusted for the covariate).



%----------------------------------------------------------%
\subsection{Post hoc tests}
\begin{itemize}
\item 
Now that you know there is a statistically significant difference between the adjusted means, you will want to know where the differences lie. This is reported in the Pairwise Comparisons table, as shown below:
\item 

By consulting the significance levels ("Sig." column), you can see which group comparisons are statistically significantly different. You can report these results in a similar manner to the one-way ANOVA, but substituting in adjusted means rather than original mean.
\end{itemize}


%----------------------------------------------------------%
\subsection{Putting it all together}
In our enhanced ANCOVA guide, we show you how to write up the results from your assumptions tests, ANCOVA and post hoc results if you need to report this in a dissertation/thesis, assignment or research report. We do this using the Harvard and APA styles. You can learn more about our enhanced content here.
