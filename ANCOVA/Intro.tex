Analysis of covariance (ANCOVA) is a general linear model which blends ANOVA and regression. ANCOVA evaluates whether population means of a dependent variable (DV) are equal across levels of a categorical independent variable (IV) often called a treatment, while statistically controlling for the effects of other continuous variables that are not of primary interest, known as covariates (CV) or nuisance variables. Mathematically, ANCOVA decomposes the variance in the DV into variance explained by the CV(s), variance explained by the categorical IV, and residual variance. Intuitively, ANCOVA can be thought of as 'adjusting' the DV by the group means of the CV(s).[1]

The ANCOVA procedure is described as follows, assuming that a linear relationship between the response (DV) and covariate (CV) exists:

{\displaystyle y_{ij}=\mu +\tau _{i}+\mathrm {B} (x_{ij}-{\overline {x_{i}}})+\epsilon _{ij}} y_{{ij}}=\mu +\tau _{i}+\mathrm{B} (x_{{ij}}-\overline {x_{i}})+\epsilon _{{ij}}

where {\displaystyle y_{ij}} y_{{ij}} is the jth observation under the ith categorical group, {\displaystyle \mu } \mu  is the grand mean, {\displaystyle \tau _{i}} \tau _{i} is the effect of the ith level of the IV, {\displaystyle x_{ij}} x_{ij} is the jth observation of the covariate under the ith group, {\displaystyle {\overline {x_{i}}}} \overline {x_{i}} is the ith group mean, and {\displaystyle \epsilon _{ij}} \epsilon _{{ij}} is the associated unobserved error term. Under this specification, we assume that the categorical treatment effects sum to zero {\displaystyle \left(\sum _{i}^{a}\tau _{i}=0\right).} \left(\sum_i^a \tau_i = 0\right). The standard assumptions of the linear regression model are also assumed to hold, as discussed below.[2]
